\documentclass{article}
\usepackage[utf8]{inputenc}
\usepackage{amsmath}
\usepackage{amssymb}
\usepackage{ragged2e}

\begin{document}

\begin{justify}

\section*{Exercice 4.3 : Démonstration de l'égalité entre l'Indice de Moran et la pente de régression du décalage spatial}

Soit $X$ un vecteur d'observations $x_1, x_2, \dots, x_n$. Soit $W$ une matrice de voisinage de dimension $n \times n$. 

\subsection*{Hypothèses}

La matrice $W$ est normalisée par ligne : pour chaque ligne $i$, $\sum_{j=1}^{n} w_{ij} = 1$.


\subsection*{1. Égalité des moyennes ($\overline{WX} = \bar{x}$)}
Pour démontrer que la pente de régression est égale à l'indice de Moran, nous devons d'abord prouver que la moyenne de la variable décalée ($WX$) est identique à celle de la variable d'origine ($X$).

\vspace{3mm}

\noindent Le décalage spatial au point $i$ est défini par : $(WX)_i = \sum_{j=1}^{n} w_{ij}x_j$.
La moyenne de ces décalages est :
$$ \overline{WX} = \frac{1}{n} \sum_{i=1}^{n} (WX)_i = \frac{1}{n} \sum_{i=1}^{n} \left( \sum_{j=1}^{n} w_{ij}x_j \right) $$

\noindent En inversant l'ordre des sommations :
$$ \overline{WX} = \frac{1}{n} \sum_{j=1}^{n} x_j \left( \sum_{i=1}^{n} w_{ij} \right) $$

\noindent Si l'on considère que la somme des poids par colonne est égale à 1 (cas d'une matrice bistochastique ou symétrique normalisée) :
$$ \overline{WX} = \frac{1}{n} \sum_{j=1}^{n} x_j (1) = \bar{x} $$

\subsection*{2. Centrage du vecteur décalé ($WX$)}

Montrons que si le vecteur $X$ est centré, le vecteur spatialement décalé $WX$ l'est également.

\vspace{3mm}

\noindent Supposons que la variable $X$ est centrée. Par définition, cela signifie que sa moyenne est nulle :
$$ \bar{x} = 0 $$

\noindent D'après la propriété démontrée dans la section 1, nous savons que les moyennes sont égales ($\overline{WX} = \bar{x}$).
\noindent En remplaçant $\bar{x}$ par 0, nous obtenons immédiatement :
$$ \overline{WX} = 0 $$

\noindent \textbf{Conclusion :} Si la matrice $W$ est normalisée et si le vecteur $X$ est centré, alors la moyenne du vecteur spatialement décalé est nulle. Le vecteur $WX$ est donc également centré.


\subsection*{3. Expression de l'Indice de Moran ($I$)}

Soit $X$ un vecteur de $n$ observations $x_1, x_2, \dots, x_n$ associées à des unités spatiales.

\vspace{3mm}
 
\noindent La formule générale de l'indice de Moran est :
$$ I = \frac{n}{S_0} \frac{\sum_{i=1}^{n} \sum_{j=1}^{n} w_{ij}(x_i - \bar{x})(x_j - \bar{x})}{\sum_{i=1}^{n} (x_i - \bar{x})^2} $$

\vspace{2mm}

\begin{center}
avec $S_0 = \sum_{i=1}^{n} \sum_{j=1}^{n} w_{ij}$
\end{center}

\vspace{3mm}

\noindent Comme $W$ est normalisée par ligne, alors $\frac{n}{S_0} = 1$. En sortant le terme $(x_i - \bar{x})$ de la somme interne sur $j$, on obtient :
$$ I = \frac{\sum_{i=1}^{n} (x_i - \bar{x}) \left[ \sum_{j=1}^{n} w_{ij}(x_j - \bar{x}) \right]}{\sum_{i=1}^{n} (x_i - \bar{x})^2} $$

\subsection*{4. Expression de I en fonction du décalage spatial ($WX$)}

Le décalage spatial au point $i$ est défini par : $(WX)_i = \sum_{j=1}^{n} w_{ij}x_j$.

\vspace{3mm}

\noindent Développons le terme entre crochets dans l'expression de $I$ :
$$ \sum_{j=1}^{n} w_{ij}(x_j - \bar{x}) = \sum_{j=1}^{n} w_{ij}x_j - \bar{x} \sum_{j=1}^{n} w_{ij} $$
Puisque $\sum_{j=1}^{n} w_{ij} = 1$, on a :
$$ \sum_{j=1}^{n} w_{ij}(x_j - \bar{x}) = (WX)_i - \bar{x} $$

\noindent En remplaçant dans la formule de $I$, nous arrivons à :
\begin{equation}
I = \frac{\sum_{i=1}^{n} (x_i - \bar{x})((WX)_i - \bar{x})}{\sum_{i=1}^{n} (x_i - \bar{x})^2}
\end{equation}

\subsection*{5. Pente de la droite de régression ($\beta$)}
Considérons la régression linéaire simple de $WX$ sur $X$ : $(WX)_i = \alpha + \beta x_i + \epsilon_i$.

\vspace{2mm}

La formule du coefficient de pente $\beta$ par les Moindres Carrés Ordinaires (MCO) est :
$$ \beta = \frac{Cov(X, WX)}{Var(X)} = \frac{\sum_{i=1}^{n} (x_i - \bar{x})((WX)_i - \overline{WX})}{\sum_{i=1}^{n} (x_i - \bar{x})^2} $$

\vspace{2mm}

\noindent Or, d'après la section 1, nous savons que $\overline{WX} = \bar{x}$. On en déduit :
\begin{equation}
\beta = \frac{\sum_{i=1}^{n} (x_i - \bar{x})((WX)_i - \bar{x})}{\sum_{i=1}^{n} (x_i - \bar{x})^2}
\end{equation}

\subsection*{Conclusion}
En comparant les équations (1) et (2), nous concluons que :
$$ I = \beta $$
L'indice de Moran global correspond donc à la pente de la droite de régression linéaire de Moran.

\end{justify}

\end{document}